% !TEX root = ../termpaper.tex
% first example for the abstact
% @author Thomas Lehmann
%
\vspace*{0.4cm}

\noindent 
In dieser Arbeit wird ein Algorithmus vorgestellt, der versucht, auf Grundlage von Constraint Satisfaction Algorithmen, ein beliebiges 
Minesweeper Spiel zu lösen. Die dabei eingesetzten Algorithmen schließen AC-3, Revise und Backtracking ein. Es werden außerdem die
Herausforderungen, die das Lösen von Minesweeper bietet, diskutiert und angegangen. Der daraus enstandene \textit{Solver} wird getestet
und zeigt eine abnehmende Effizienz bei steigender Komplexität der Spielfeldkonfiguration (bestehend aus Spalten, Zeilen und Anzahl Minen).
Im Vergleich zu anderen Implementierungen und Ansätzen für Minesweeper schneidet er schlechter ab, ist jedoch auch im Vergleich zu anderen
Lösungsalgorithmen einfacher aufgebaut und hat die Möglichkeit weiter ausgebaut zu werden.