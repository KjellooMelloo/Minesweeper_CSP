% !TEX root = ../termpaper.tex
% @author Kjell May
%

\section{Algorithmus}

\subsection*{Definition}
Um ein Constraint Satisfaction Problem formal beschreiben zu können, müssen eine Menge von Variablen, die Menge 
ihrer Wertebereiche und die Definition der Constraints festgelegt werden. Für Minesweeper habe ich dies wie folgt definiert:
Für ein Spielfeld mit $n\in \mathbb{N}$ Spalten und $m\in\mathbb{N}$ Zeilen sind
die Variablen $X=\{(x, y) | 2\le x\le n  \text{ und }  2\le y\le m\}$ und ihre Wertebereiche $D=\{0, 1\}$, wobei 1
für eine Mine steht und 0 für ein sicheres Feld.
Die Constraints müssen umfangreicher definiert werden. Für jede Variable $(x, y)$ gibt es den Constraint auf ihm und alle seiner 
Nachbarn, dass der Wert auf dem Feld der Variablen, desweiteren als Konstante $k(x, y) \text{ mit } k\in\mathbb{N}$ 
benannt, gleich der Summe aller Werte der Nachbarsvariablen ist. Formal definiert wäre dies
$C=\{C(x, y) | k(x, y) = \sum_{\substack{0\le a\le n-1\\0<b<m-1}}{(a, b)} \text{ mit } (a, b)\in N_G((x, y))\}$.
Binäre Constraints zwischen zwei benachbarten Variablen ergeben sich dann dadurch, dass eine Variable keinen Wert annehmen kann, 
welcher den Constraint des Nachbars verletzen würde. Diese sind Teilschritte dahin, den eigentlichen Constraint erfüllen zu können.
Der Constraint-Graph ergibt sich für dieses Spiel trivial als Graph des Spielfelds.

\subsection*{Besonderheiten Minesweeper}
Eine der Besonderheiten, um das Spiel Minesweeper zu lösen zu können, ist das Aufdecken der Felder, womit die Konstante auf diesem Feld
bekannt wird und daraus dynamisch Constraints definiert werden können. Daraus ergibt sich, dass der Algorithmus immer wiederholt wird,
wenn neue Informationen durch Aufdecken erlangt werden. Mit dieser Art der unvollständigen Information über das Spielfeld unterscheidet es
sich deutlich von anderen CSPs wie Sudoku, das Vier-Farben-Problem oder das Damenproblem.
Eine weitere Herausforderung ist, dass nicht jedes Spiel mit Sicherheit lösbar ist. Es gibt Fälle, in denen nur mit 50\% Wahrscheinlichkeit
gesagt werden kann, wo sich eine Mine befindet. Beispiel siehe Abb. A

\subsection*{Ablauf}
\subparagraph*{1. Startpunkt und Aufdecken}

Die erste Herausforderung zum Lösen dieses Spiels ist die Wahl, welches Feld zuerst aufgedeckt wird, da alle Felder zu Beginn verdeckt sind.
In der Windows-Version von Minesweeper ist sichergestellt, dass der erste Klick keine Mine aufdecken kann (Quelle X). Dies habe ich für 
meine Implementierung des Spiels auch übernommen, aus einfachen Komfortgründen. Ist diese Voraussetzung gegeben, kann zu Beginn also einfach
ein zufälliges Feld gewählt werden.

Jedes Mal, wenn ein Feld aufgedeckt wird und es keine Mine ist, kann der Wertebereich der Variable für
das Feld auf $D=\{0\}$ reduziert, damit auch der Wert auf 0 gesetzt und die binären Constraints dieser Variable zu allen Nachbarn in beide
Richtungen definiert werden. Als Heuristik habe ich hier zusätzlich eingebaut, dass falls die aufgedeckte Konstante eine 0 ist, direkt
alle Nachbarn aufgedeckt werden können (rekursiv für weitere Nullen). Damit werden die trivialen Fälle der 0 direkt abgehandelt, womit
die zugehörigen Variablen bereits konsistent sind.

Im Laufe des Algorithmus ergeben sich weitere sichere Felder, die wie oben beschrieben wieder aufgedeckt werden können.
