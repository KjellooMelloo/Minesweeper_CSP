% !TEX root = ../termpaper.tex
% @author Kjell May
%
\graphicspath{{chapters/images/}}
\section{Fazit und Aussicht}

In dieser Hausarbeit wurde ein Algorithmus entwickelt und vorgestellt, der versucht Minesweeper als Constraint Satisfaction Problem lösen
zu können. Es hat sich gezeigt, dass diese Version schnell an Grenzen stößt und nur sehr einfache Konfigurationen lösen kann. Der Solver
ließe sich an einigen Stellen sicherlich noch erweitern und verbessern. Man könnte beispielsweise bei Auswahl des nächsten Feldes als 
Heuristik Wahrscheinlichkeiten, dass ein Feld eine Mine ist, nehmen, um möglicherweise mehr Fortschritt erlangen und ein vorzeitiges
Ende verzögern zu können. Ein paar Beispiel für andere Ansätze und Untersuchungen:\\
David Becerra hat in seiner Bachelorthesis \cite{AlgoApproaches} intensiver verschiedene Ansätze ausprobiert und
ist zu deutlich besseren Ergebnissen gekommen. Der Benutzer \textit{DavidNHill} hat einen Solver geschrieben (siehe \cite{statistics}), der
bei jedem Klick die Wahrscheinlichkeit für ein sicheres Feld berechnet. Robert Massaioli hat einen Algorithmus geschrieben, der Minesweeper
mit einem Matrizen-Ansatz löst (siehe \cite{matrix}). Aus einer Kombination der verschiedenen Ansätze könnte ein womöglich
effizienterer Algorithmus entwickelt werden.
Wie jedoch in \cite{MS-NP} und \cite{MS-Conf} von Richard Kaye gezeigt wurde, ist Minesweeper NP-complete und damit
ein effizienter Algorithmus zum Lösen dieses Spiel zu finden, sehr herausfordernd.
